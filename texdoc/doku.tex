\documentclass[a4paper, DIV=12, firstfoot=false]{scrreprt}
\usepackage{scrlayer-scrpage}

% Umlaute und Font-Encoding
\usepackage[T1]{fontenc}
\usepackage[utf8]{inputenc}

% Deutsch
\usepackage[ngerman]{babel}
\usepackage{csquotes}

% Das muss fucking nochmal GEIL aussehen!
% \usepackage{microtype} - Erst im Final actimelisieren

% Mathe
\usepackage{mathtools}
\usepackage{amssymb}
\usepackage{ziffer}
\usepackage{hhline}

% Bilder
\usepackage{graphicx}
\graphicspath{{Grafiken/}}
\usepackage{wrapfig}
\usepackage{tikz}
\usetikzlibrary{graphs}

% Coole Links im PDF
\usepackage[hidelinks]{hyperref}

% Aliase zum abdrucken von Shell-CMDs
\newcommand{\shellcmd}[1]{\texttt{\$ #1}\\}
\newcommand{\shellout}[1]{\texttt{#1}\\}

% Alias für Aufgabenname
\newcommand{\task}[1]{Rosinen picken}

\ihead{\task} \ohead{Laurenz Grote, Teilnahme 6745 (Team 00001)}
\renewcommand{\chapterpagestyle}{scrheadings}
\pagestyle{scrheadings}

\begin{document}
	\titlehead{Teilnahme 6745 (Team 00001) \hfill Laurenz Grote}
	\title{\task}
	\subtitle{Aufgabe 1}
	\author{Laurenz Friedrich Grote}
	\date{}
	\maketitle
	\tableofcontents
	\vspace {2em}
	Meine Umsetzung für "`\task{}"' erfolgte unter Linux mit Java (OpenJDK 1.8). Ich habe Ihnen den Programmcode sowie eine ausführbare .jar-Datei beigelegt.
	\pagebreak
	% ----------------------------------------------------------------------------
	\chapter{Lösungsidee}
		Zunächst habe ich versucht, die Anwendung aus der Aufgabenstellung in eine formale Problemstellung zu übersetzen:
\begin{displayquote}
	Gegeben ist ein ein gerichteter Graph \(G=(V,E)\)\footnote{\(V_{ertices}\) ist die Menge der Knoten (Firmen), \(E_{dges}\) die Menge der Kanten (Nebenbedingungen)} mit gewichteten Knoten\footnote{Das Gewicht eines Knotens entspricht dem Kaufpreis}. Gesucht wird die Menge \(K \in V\), bei der die Summe der Gewichte der enthaltenen Knoten maximal ist. Bei der Aufnahme eines Knotens  \(A\) in die Menge \(K\) müssen ebenfalls alle Knoten, zu denen von \(A\) ein gerichteter Pfad führt, aufgenommen werden.
\end{displayquote}

Danach habe ich für die Auswahl von geeigneten Algorithmen die Eigenschaften des Graphens untersucht:
\begin{itemize}
	\item Der Graph ist gerichtet
	\item Parallele Kanten ergeben im Kontext der Aufgabenstellung keinen Sinn, da jede Firma nur einmal erworben werden kann
	\item Reflexive Kanten ergeben ebenfalls keinen Sinn, da eine Firma sich nicht selbst vorraussetzen kann
	\item Zyklen können vorkommen
	\item Nicht jeder Knoten besitzt notwendigerweise Kanten, da einige Firmen ohne Nebenbedingungen erwerbbar sind
	\item Der Graph hat daher beliebig viele starke wie schwache Zusammenhängigkeitskomponenten
	\item Der Zahlenraum für die Knotengewichte geht nicht aus der Aufgabenstellung hervor. Logischerweise hat eine Firma jedoch einen beliebigen positiven oder negativen Wert. Daher liegt das Gewicht eines jeden Knotens in \(\mathbb{R}\).
\end{itemize}

Mit diesen Informationen habe ich einen Algorithmus entwickelt, der die optimale Teilmenge ermittelt:

In einem ersten Schritt werden für jede Firma des Konglomerates die Firmen ermittelt, die aufgrund der Nebenbedingungen zwingend mit erworben werden müssen. Im Kontext des Graphens bedeutet dies, dass zu jedem Knoten \(A \in V\) die Menge der Knoten, zu denen von \(A\) ein Pfad führt, ermittelt wird. Dieser Menge wird der Ursprungsknoten hinzugefügt. Diese Menge hat wie auch ein Knoten ein Gewicht, dass sich aus der Summe der Gewichte der enthaltenen Knoten zusammensetzt. Jede dieser Mengen stellt einen Kauf einer Firma des Konglomerates dar, bei dem alle Nebenbedingungen berücksichtigt werden.

Allerdings gibt es darüber hinaus noch weitere gültige Käufe/Teilmengen. Denn logischerweise können in einer Transakation beliebig viele Firmen des Konglomerates erworben werden. Daher stellt zum Beispiel auch die Menge bestehend aus den Knoten \((A, B) \in V\) und den Knoten, zu denen ein gerichteter Pfad von \(A\) oder \(B\) existiert, eine gültige Transaktion im Sinne der Aufgabenstellung dar. Denn, um wieder in den Kontext der Aufgabenstellung zurückzukehren, kann der Käufer eine Firma entweder unbedingt haben wollen oder den Kauf einer Firma für optional halten. Verallgemeinert betrachtet bedeudet dies also, dass es bis zu \(2^V\) gültige Teilmengen gibt.

\begin{wrapfigure}{r}{0.4\textwidth}
  \begin{center}
    \begin{tikzpicture}[scale=0.75]
	\graph [nodes={draw, circle}, no placement] {
		F[x=3, y=0] -> {R[x=0, y=2], E[x=6, y=2]};
		Z[x=5, y=4] -> {A[x=3,y=4], B[x=4, y=2]};
		A -> {C[x=3, y=6] -> {D[x=0, y=4], R}};
		B -> {E,I[x=2, y=2] -> {R}};	
	};
\end{tikzpicture}
  \end{center}
  \caption{Beispielgraph}
  \label{abb:vereinigung}
\end{wrapfigure}


Die Bestimmung der aufgrund von Nebenbedingungen mitzukaufenden Firmen für \(2^V\) mögliche Kaufwünsche wird bei großer Knotenanzahl sehr zeitaufwändig, schließlich handelt es sich um exponentielles Wachstum. Allerdings lassen sich diese Mengen auch durch Vereinigung der Mengen aus dem ersten Berechnungsschritt ermitteln. Wenn Beispielsweise zum Kauf von \(A\) der Erwerb von \(M_A=\{A, C, D, R\}\) und zum Kauf von \(B\) der Erwerb von \(M_B = \{B, I, E, R\}\) erforderlich ist, ist für den Kauf von \(A\) und \(B\) der Erwerb der Menge \(M_A \cup M_B = \{A, B, C, D, E, I, R\}\) erforderlich (siehe Abbildung \ref{abb:vereinigung}). Durch die Ausnutzung dieser Eigenschaft lassen sich die \(2^V\) Mengen wesentlich schneller bestimmen, denn alle Mengen auf der linken Seite sind aus dem ersten Berechnungsschritt schon bekannt.

Die Menge mit dem höchsten Gewicht (=Wert) entspricht der Teilmenge des Graphens mit dem höchsten Wert.
Die Mengen, die aufgestellt werden müssen, lassen sich dank folgender Beobachtung erheblich reduzieren:
Knoten mit einem negativen Gewicht ergeben nur als Nebenbedingung Sinn. Denn ein ökonomisch sinnvoll handelnder Käufer würde niemals eine Firma mit einem negativen Wert freiwillig kaufen wollen. Das bedeutet, dass bei einer optimalen Auswahl auf der linken Seite, also den nicht als Nebenbedingung gekauften Firmen, nur Firmen/Knoten mit einem positiven Gewicht stehen können. Firmen ohne Wert ergeben ebenfalls keinen ökonomischen Mehrwert und könnene daher ebenfalls nicht auf der linken Seite stehen. Damit sinkt die Anzahl der zu beachtenden Teilmengen auf \(2^p\), wobei p der Anzahl der Knoten mit positivem Gewicht entspricht.

Die Anzahl der zu beachtenden Mengen kann jedoch noch weiter reduziert werden: Es gibt keinen Grund, eine positivwertige Firma nicht zu kaufen, die nur positivwertige Firmen als Nebenbedingung hat. Bei solchen Teilmengen entsteht ausschließlich Gewinn. Daher können alle Mengen aus ausschließlich positivwertigen Firmen zu einer "`Win-Win-Menge"' vereinigt werden. Diese Menge wird nun mit den übrig gebliebenen Mengen vereinigt, da sie ja im besten Kauf enthalten sein muss. Die Anzahl der zu berücksichtigenden Mengen ist nun auf \(2^{pm}\) reduziert. \(pm\) entspricht den positivwertigen Firmen, die negativwertige Nebenbedingungen haben.
Damit ist jedoch nicht das Grundproblem des exponentiellen Wachstums beseitigt. Je nach verfügbarem Arbeitspeicher wird die Zustandsmenge bei einem hinreichend großem pm irgendwann so groß, dass keine weiteren Zustände mehr gespeichert werden könnnen.
Für diesen Fall muss dem Algorithmus ein heuristisches Element hinzugefügt werden. Sobald dieses eingesetzt wird, wird nicht mehr mit absoluter Sicherheit die beste Teilmenge gefunden, da Zwischenergebnisse, von denen nicht erwartet wird, dass sie Teil der besten Teilmenge sind, zur Einsparung von Speicherplatz verworfen werden.
	\chapter{Umsetzung}
		\section{Datenstruktur der Firmen und des Firmenkonglomerates}
Jede Firma wird in meiner Implementierung durch ein Objekt abgebildet, welches neben einer ID und dem Wert als Double auch eine Liste über alle unmittelbaren Nebenbedingungen enthält. Ich habe mich somit für die gängige Implementierung der Graphstruktur als Adjazenzliste entschiden. Da die einzige Operation auf dem Graphen eine Tiefensuche ist, ist die Adjazenzliste die geeigneteste Implementierung.\footnote{vgl. Sedgewick, Algorithmen: S. 563 (Deutsche Übersetzung der 4. Auflage)}

Das Firmenkonglomerat speichert alle Firmen des Konglomerates als Array nach Firmen-ID indiziert. Darüber hinaus speichert es die Nebenbedingungsmengen zu jeder Firma wie auch die Information, welche Nebenbedingungsmengen ausschließlich positivwertige Firmen beinhalten, in weiteren Arrays ab. Die Konglomeratsklasse verfügt über Methoden zur Bestimmung der Nebenbedingungsmengen und der idealen Teilmenge. Die Datenstrukturen finden sich in den Dateien Company.java und und Conglomerate.java.

\section{Bestimmen der Nebenbedingungen jeder Firma (Listing \ref{lst:abn})}
Die Menge aller Nebenbedingungen eines Feldes \(A\) ist äquivalent zu den besuchten Knoten einer Tiefensuche von diesem Feld \(A\) aus. Schließlich ist es das Funktionsprinzip einer Tiefensuche, alle Felder abzusuchen, zu denen ein Pfad von einem Feld führt. Eine Tiefensuche funktioniert, indem zunächst alle angrenzenden Knoten des Feldes \(A\) (die Nebenbedingungen) auf einen Stapel gespeichert werden.
Außerdem wird \(A\) der Besucht-Menge hinzugefügt. Alle Felder auf dem Stapel werden nun ebenfalls der Besucht-Menge hinzugefügt. Die an diese Felder angrenzenden Felder werden widerum auf den Stapel gelegt, sodenn sie nicht schon in der Besucht-Menge liegen. So wird jedes Feld, zu dem von \(A\) ein Pfad führt, in die Besucht-Menge aufgenommen. Diese Besucht-Menge entspricht daher der Teilmenge von \(G\), die beim Kauf von \(A\) mitgekauft werden muss. Der Algorithmus kann sich nicht in einem Zyklus verfangen, da wiederholte Besuche eines Feldes mit der Besucht-Menge verhindert werden. Die Nebenbedingungsmengen für jede Firma speichere ich in einem Array der Firmenkonglomeratsklasse.

Bei der Tiefensuche ermittele ich außerdem, ob eine negativwertige Firma in der Nebenbedingungsmenge liegt. Dies ist, wie in der Lösungsidee beschrieben, zur Beschleunigung des Algorithmusses relevant. Ob eine negativwertige Firma in der Ergebnismenge enthalten ist, speichere ich einem Array.

Zentral für die Umsetzung der Lösungsidee ist die effiziente Abbildung der Besucht-Menge im Speicher. Naheliegend ist die Verwendung der Klasse Set, jedoch ist diese bei genauerer Betrachtung nicht optimal. Sinnvoller ist die Verwendung eines BitSets. Ein BitSet besteht in Java aus einer Abfolge von \(n\) binären Bits. Für unsere Anwendung wird für jede Firma ein Bit benötigt.
Steht das Bit \textit{n} auf 1, befindet sich die \textit{n}te Firma in der durch das BitSet repräsentierte Menge. Steht es auf 0, befindet sich die Firma nicht in der Menge.
Die Verknüpfung zweier Mengen entspricht bei einem Bitset einer nicht-exklusiven ODER-Verbindung (siehe Tabelle \ref{tab:veroder} zu Abbildung \ref{abb:vereinigung}).
Das BitSet ist besser geeignet, da anstatt der jeweiligen Firmennummern als 32bit-Integer nur einzelne Bits abgespeichert werden müssen. Außerdem gelingt die Vereinigung zweier Firmen schneller, da keine Zahlen verglichen werden müssen. Ein weiterer Vorteil des BitSets ist, dass es wahlfreien Zugriff unterstützt, während ein Set in der Standardimplementierung nur über Iteratoren auslesbar ist.
\begin{table}
	\centering
    \begin{tabular}{l|ccccccccc}
    Firma 	& A & B & C & D & E & F & I & R & Z \\ \hline
    \(M_A\)	& 1 & 0 & 1 & 1 & 0 & 0 & 0 & 1 & 0 \\
    \(M_B\)	& 0 & 1 & 0 & 0 & 1 & 0 & 1 & 1 & 0 \\ \hhline{=|*{9}{=}}
    ODER 	& 1 & 1 & 1 & 1 & 1 & 0 & 1 & 1 & 0 \\
    \end{tabular}
    \caption {Vereinigung entspricht ODER}
    \label{tab:veroder}
\end{table}

\section{Aufstellen des Baumes (Listing \ref{lst:teilmenge})}
Für die Repräsentation des Baumes habe ich aufgrund der spezifischen Anforderungen des Algoritmusses einen neue Datenstruktur, eine \textit{BufferedMap}, implementiert. Diese ist aus vielerlei Gründen besser geeignet als ein Array oder eine auf Referenzen basierende Struktur.
Eine BufferedMap besteht aus einer Map, auf die nur lesend zugegeriffen werden kann, und einem Puffer, auf den wiederum nur schreibend zugegriffen werden kann. Neue Einträge werden zunächst auf den Puffer geschrieben und zu einem späteren Zeitpunkt gesammelt der Map hinzugefügt. Der Puffer ist als Stack implementiert. Die Map hat als Schlüssel eine Menge, also ein BitSet, und als Wert den Wert dieser Menge als Double.

Ein Vorteil dieser Datenstruktur ist, dass auf den Puffer geschrieben werden kann, während gleichzeitig über die Map iteriert wird. Der Iterator der Map bleibt stabil, die während der Iteration geschriebenen Daten werden während der weiteren Iteration nicht ausgelesen. Erst nach der Iteration werden sie dem Set über den gesammelten Schreibvorgang hinzugefügt.

Außerdem werden dank der durch die Map gegebene Eindeutigkeit der Keys nicht \(2^pm\) Blätter aufgestellt, da viele Blätter des Baumes nach Betrachtung von Nebenbedingungen äquivalent sind. Im Beispiel aus Abbildung \ref{abb:vereinigung} sind beispielsweise die Mengen von den Kaufwünschen \(B\) und \(B, E\) nach Einbeziehung der Nebenbedingungen äquivalent, da \(B\) \(E\) als Nebenbedingung hat. Solche Dopplungen werden von der Map durch Vergleich der HashCodes automatisch herausgefiltert.

In einem ersten Schritt fasse ich alle Nebenbedingungsmengen, die während der Tiefensuche als "`Win-Win-Menge"' identifiziert wurden, zu eben dieser "`Win-Win-Menge"' zusammen. Diese Menge, wie oben beschrieben repräsentiert durch ein BitSet, füge ich nun als erstes Element der BufferedMap hinzu.
Darüber hinaus speichere ich in zwei seperaten Variablen den Wert und alle Firmen der "`Win-Win-Menge"'. Diese Variablen dienen als Speicher für den bis dato besten gefundene Kauf. Sobald der Baum komplett aufgebaut ist, findet sich in diesen Variablen der beste Kauf des Konglomerates.

Danach ermittele ich die Liste pm, also die Firmen, die Teil des Kaufwünsches für den besten Kauf sein könnten. Dies geschieht, indem ich alle Nebenbedingungsmengen, deren Ursprungsfirma einen positiven Wert hat und noch nicht in der "`Win-Win-Menge"' enthalten sind, in diese Liste hereinkopiere. Die Liste ist als HashSet implementiert um mögliche Doppelungen herauszufiltern. Nun führe ich folgende Logik für alle Nebenbedingungsmengen aus der Liste pm aus:

Ich füge der BufferedMap alle Kombinationen aus den bisher errechneten Käufen und einer Nebenbedingungsmenge aus pm hinzu. Die geschieht, indem ich über die gesamte Map iteriere und jeden Mapeintrag mit der Nebenbedingungsmenge verknüpfe. Den daraus resultierenden Eintrag lege ich auf den Puffer, da die gleiche Firma ja nicht zweimal in einem Eintrag kombiniert werden kann. Wenn die Menge mit allen Mapeiträgen verknüpft ist, schreibe ich den Puffer und fahre mit der nächsten Nebenbedingungsmenge fort. Damit erhalte ich alle Blätter des Baumes. Durch diese Implemenentation des Baumes verliere ich alle Generationen bis auf die letzte. Da allerdings alle Käufe in den Blättern vorhanden sind, geht keine relevante Information verloren. Falls dabei eine Menge gefunden wird, deren Wert höher als das bisherige Maximum ist, speichere ich diese Menge und ihren Wert in den entsprechenden Maximumsvariablen. So finde ich mit absoluter Sicherheit die günstigste Kombination. Den Zuwachs der Map von Schleifendurchlauf zu Schleifendurchlauf habe ich in Abbildung \ref{abb:map} skizziert.

\begin{figure}[!h]
	\centering
	\begin{tikzpicture}[every node/.style={draw,circle, minimum size=2cm}, scale=0.75]
	\node at (0, 0) (W) 	{\tiny WinWin};
	\node at (3, 0) (WA)	{\tiny WinWin,A};
	\node at (6, 0) (WAB)	{\tiny WinWin,A,B};
	\node at (6, 3) (WB)	{\tiny WinWin,B};
	\node at (9, 3) (WABC)	{\tiny WinWin,A,B,C};
	\node at (9, 0) (WAC)	{\tiny WinWin,A,C};
	\node at (9, 6) (WBC)	{\tiny WinWin,B,C};
	\node at (9, 9) (WC)	{\tiny WinWin,C};

	\draw[dashed] (-1.5, -1.5) -- (-1.5, 10.5);
	\draw[dashed] (1.5, -1.5) -- (1.5, 10.5);
	\draw[dashed] (4.5, -1.5) -- (4.5, 10.5);
	\draw[dashed] (7.5, -1.5) -- (7.5, 10.5);
	\draw[dashed] (10.5, -1.5) -- (10.5, 10.5);

	\draw [->] (W) edge[bend left=90] (WA) (W) edge[bend left=64] (WB) (W) edge[bend left = 45] (WC);
	\draw [->] (WA) edge[bend right=90] (WAB) (WA) edge[bend right=90] (WAC);
	\draw [->] (WAB) edge (WABC);
	\draw [->] (WB) edge (WBC);
\end{tikzpicture}
	\caption{Wachstum der Map}
	\label{abb:map}
\end{figure}

\section{Implementierung der Heuristik (Listing \ref{lst:buffmap})}
Die Heuristik habe ich innerhalb der BufferedMap implementiert. Nach jeder Leerung des Puffers wird überprüft, ob die maximale Größe der Map überschritten wurde.
Sollte dies der Fall sein, werden niedrigwertige Knoten aus dem Baum gelöscht. Dafür ist es nötig, die Knoten nach Wert zu sortieren.

Da die Standardbibliothek eine solche Sortierung nicht vorsieht, musste ich einen eigenen Komparator (Listing \ref{lst:komp}) entwickeln. Dieser Komparator besitzt eine Referenz auf die Map. Beim Vergleich von zwei Mengen lädt der Komparator ihre jeweiligen Werte aus der Map und vergleicht diese. Der Komparator ist so ausgeführt, dass aufsteigend sortiert wird.

Für die Sortierung wird ein Array über alle Keys, also die Knoten des Baumes, aus der Map extrahiert. Dieses wird dann mithilfe einer parallelen MergeSort-Implementierung sortiert. MergeSort hat den Vorteil das die Laufzeit vorhersehbar ist und somit der Nutzer vor Ablauf geeignete Parameter auswählen kann. Aufgrund der großen Datenmenge und einer fertigen Implementierung innerhalb der Standardbibiliothek lohnt sich Parallelisierung an dieser Stelle.

Anschließend werden soviele Einträge wie vom Nutzer spezifiziert gelöscht. Um im Nachhinein festzustellen, ob die Heurisitk angewandt wurde, wird nach jedem Heuristikdurchlauf ein Zähler erhöht.

Da die Heuristik innerhalb der Datenstruktur implementiert wurde, ist sie für den restlichen Algorithmus transparent und kann gegebenfalls leicht durch eine bessere ersetzt werden.
		\section{Überlegungen zum Laufzeitverhalten}
			Relevant für die Beurteilung derGesamtlaufzeit sind der Baumaufbau und etwaige Heuristikdurchläufe.

Wie schon in der Lösungsidee geschildert, gibt es bei meinem Programm zwei Modi. Solange eine maximale Baumgröße nicht überschritten wird, wird der Baum komplett aufgebaut.
Da sich die Anzahl der Elemente im Baum, die mit anderen Elementen verknüpft werden müssen und von denen der Wert berechnet werden muss, sich bei jedem Durchlauf verdoppelt, gilt für diesen Codeteil eine Komplexität von \(\mathcal{O}(2^{pm})\). \(pm\) ist hierbei wie in der Lösungsidee definiert. 

Sobald \(2^{pm}\) allerdings die obere Schranke der maximalen Elementezahl überschreitet, gilt folgende Laufzeitkomplexität für den Baumaufbauteil:

\begin{gather}
	{hHeur} = max - \frac{p}{100} \times max \\
	{nHeur} = \frac{V-max}{max} \\
	{rem}   = h-max \bmod max \\
	\mathcal{O}(max+nHeur\times{}(max-2^{hHeur})+(2^{hHeur + rem}-2^{rem}))
\end{gather}

Zunächst wird ein Baum von der größe Max aufgebaut. Danach wird so oft die Heuristik aufgerufen, bis alle Firmen berücksichtigt wurden. Pro Heuristikdurchlauf werden soviele Knoten hinzugefügt, wie die Differenz zwischen dem nächsten Heuristikdurchlauf (max) und den übrig gelassenen Knoten beträgt. Schlussendlich wird noch der Teilbaum für den Rest nach dem letzten Heuristikdurchlauf aufgebaut.

Hinzu kommt noch die Laufzeit für den Sortieralgorithmus. TODODESTODES
	\chapter{Beispiele}
		Zur Einsparung von Druckseiten gebe ich hier nur Knotenzahl, Wert und ob die Heuristik angewandt wurde tabellarisch wieder. Komplette Ausgaben finden Sie im Einsendungszip. Die Rechenzeit belief sich auf einem Intel Core i5 6500 (3,2GHz) außer bei Quadrat 13 auf circa 0,2 Sekunden je Beispiel. Quadrat 13 benötigte 1,3 Sekunden.

\begin{table}[!h]
	\centering
    \begin{tabular}{lllll}
    Name       & Quadrat 6 	& Quadrat 8 				& Quadrat 13      	& Kreis 8		\\ \hline
    Firmenzahl & 25        	& 28        				& 78              	& 8      		\\
    Wert       & 333       	& 446       				& 961             	& 4      		\\
    Heuristik  & Nein      	& Nein      				& Ja (10 Aufrufe) 	& Nein  		\\ \\
    Name       & Zufall 7 	& Zufall 7 (No Single) 		& Zufall 40 		& Zufall 100 	\\ \hline
    Firmenzahl & 0       	& 0                    		& 31        		& 21        	\\
    Wert       & 0        	& 0                    		& 504       		& 613        	\\
    Heuristik  & Nein     	& Nein                 		& Nein      		& Nein      	\\
    \end{tabular}
    \caption{Ergebnisse der BwInf-Beispieleingaben}
\end{table}
\end{document}
