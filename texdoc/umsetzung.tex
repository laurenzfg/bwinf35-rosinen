Zentral für die Umsetzung der Lösungsidee ist die effiziente Abbildung der Mengen im Speicher. Naheliegend ist die Verwendung der Klasse Set, jedoch ist diese bei genauerer Betrachtung nicht optimal. Sinnvoller ist die Verwendung eines BitSets. Ein BitSet besteht in Java aus einer Abfolge von \(n\) binären Bits. Für unsere Anwendung werden soviele Bits benötigt wie der Graph Knoten hat. Jedes Bit repräsenteirt hierbei eine Firma. Steht das Bit \textit{n} auf 1, befindet sich die \textit{n}te Firma in der durch das BitSet repräsentierte Teilmenge, steht es auf 0 befindet sich die Firma nicht in der Teilmenge. Die Verknüpfung zweier Mengen lässt sich durch eine nicht-exklusive ODER-Verbindung durchführen. Das BitSet ist besser geinget, da anstatt der jeweiligen Firmennummern nur Bits abgespeichert werden müssen und die Vereinigung zweier Firmen schneller gelingt. Ein weitere Vorteil ist das das BitSet wahlfreien Zugriff unterstützt, während ein Set nur über Iteratoren auslesbar ist.